\section{Algorithmic Components: DRP}


\subsection{Instrument Signature Removal}
\begin{itemize}
\item All the usual stuff (bias, dark, linearity, crosstalk, fringing, ...?)
\item Includes B-F correction.
\item Does not include frozen-in coordinate distortions.
\item Flat-field to sky SED.
\end{itemize}

\subsection{Artifact Detection}
\subsubsection{Single-Exposure Morphology}
\begin{itemize}
\item Find CRs via morphology.
\item Find satellites via Hough transform.
\item Find some optical ghosts (etc?) from bright star catalog and optics predictions.
\end{itemize}
\subsubsection{Snap Subtraction}
\begin{itemize}
\item All of the above, but improve by looking at both snaps.
\end{itemize}
\subsubsection{Warped Image Comparison}
\begin{itemize}
\item Find more optical artifacts by looking at differences between warped images (this is run during background matching).
\item Find transient astronomical sources we don't want to include in coadds.
\end{itemize}

\subsection{Artifact Masking/Interpolation}
\begin{itemize}
\item Set mask planes for all artifacts.
\item Eliminate small artifacts by interpolating them.
\end{itemize}

\subsection{Source Detection}
\begin{itemize}
\item Detect above-threshold regions and peaks in direct or difference images.
\end{itemize}

\subsection{Deblending}
\subsubsection{Single Visit Deblending}
\begin{itemize}
\item Generate HeavyFootprint deblends using only a single image.
\end{itemize}
\subsubsection{Multi-Coadd Deblending}
\begin{itemize}
\item Generate consistent HeavyFootprint deblends from coadds over multiple bands and possibly epoch ranges.
\end{itemize}

\subsection{Measurement}
\subsubsection{Variants}
\begin{itemize}
\item Single Visit Measurement
\item Multi-Coadd Measurement
\item Forced Measurement
\item Difference Image Measurement
\item Multi-Epoch Measurement
\end{itemize}
\subsubsection{Algorithms}
\begin{itemize}
\item Centroids
\item Second-Moment Shapes
\item Aperture Photometry
\item Kron Apertures
\item Petrosian Apertures
\item Galaxy Models
\item Moving Star Models
\item Trailed Point Source Models
\item Dipole Models
\end{itemize}
\subsubsection{Blended Measurement}
\begin{itemize}
\item Deblend Template Projection
\item Neighbor Noise Replacement
\item Simultaneous Fitting
\item Hybrid Models
\end{itemize}

\begin{note}[Measurement Algorithms Table]
Matrix of measurement algorithms and the contexts in which they're run, indicating which combinations are supported
\end{note}

\subsection{Background Estimation}
\begin{itemize}
\item Fit or interpolate large-scale variations while masking out detections.
\item Needs to work in crowded fields.
\item Needs to work on both difference images and direct images.
\item Need to be able to compose backgrounds measured in different coordinate systems on different scales.
\end{itemize}

\subsection{PSF Estimation}
\subsubsection{Single CCD PSF Estimation}
\begin{itemize}
\item Fit simple empirical PSF model to stars from a single exposure.
\item No chromaticity.
\item May use external star catalog, but doesn't rely on one.
\item Used in Alert Production and BootstrapImChar in DRP.
\end{itemize}
\subsubsection{Full Visit PSF Estimation}
\begin{itemize}
\item Decompose PSF into optical + atmosphere.
\item Constrain model with stars, telemetry, and wavefront data.
\item Wavelength-dependent.
\item Used in RefineImChar in DRP.
\item Must include some approach to dealing with wings of bright stars.
\end{itemize}

\subsection{Aperture Correction}
\begin{itemize}
\item Measure curves of growth from bright stars.
\item Correct various flux measurements to infinite.
\item Propagate uncertainty in aperture correction to corrected fluxes; covariance is tricky.
\end{itemize}

\subsection{Astrometric Calibration}
\subsubsection{Single Visit}
\begin{itemize}
\item Fit multi-component WCS to all CCDs in a single visit simultaneously after matching to reference catalog.
\end{itemize}
\subsubsection{Joint Multi-Visit}
\begin{itemize}
\item Fit multi-component WCS to all CCDs from multiple visits simultaneously after matching to reference catalog.
\end{itemize}

\subsection{Photometric Calibration}
\subsubsection{Single Visit}
\begin{itemize}
\item Fit zeropoint (and some small spatial variation?) to all CCDs simultaneously after matching to reference catalog.
\item Need for chromatic dependence unclear; probably driven by AP.
\end{itemize}
\subsubsection{Joint Multi-Visit}
\begin{itemize}
\item Derive SEDs for calibration stars from colors and reference catalog classifications.
\item Utilize additional information from wavelenth dependent photometric calibration built by calibration products production.
\item Fit zeropoint and possibly perturbations to all CCDs on multiple visits simultaneously after matching to reference catalog..
\end{itemize}

\subsection{PSF Matching}
\subsubsection{Image Subtraction}
\begin{itemize}
\item Match template image to science image, as in Alert Production and DRP Difference Image processing.
\end{itemize}
\subsubsection{PSF Homogeniziation for Coaddition}
\begin{itemize}
\item Match science image to predetermined analytic PSF, as in PSF-matched coaddition.
\end{itemize}

\subsection{Image Warping}
\subsubsection{Oversampled Images}
\begin{itemize}
\item Just use Lanczos.
\end{itemize}
\subsubsection{Undersampled Images}
\begin{itemize}
\item Can use PSF model as interpolant if we also want to convolve with PSF (as in likelihood coadds).  Otherwise impossible?
\end{itemize}
\subsubsection{Irregularly-Sampled Images}
\begin{itemize}
\item Approximate procedure for fixing small-scale distortions in pixel grid.
\end{itemize}

\subsection{Image Coaddition}
\begin{itemize}
\item Can do outlier rejection (but usually doesn't).
\item Needs to propagate full uncertainty somehow.
\item May need to propagate larger-scale per-exposure masks to get right PSF model or other coadded quantities.
\end{itemize}

\subsection{Star/Galaxy Classification}
\subsubsection{Single Visit S/G, Pre-PSF}
\subsubsection{Single Visit S/G, Post-PSF}
\subsubsection{Multi-Source S/G}
\subsubsection{Object S/G Classification}

\subsection{Variability Classification}

\subsection{Association and Matching}
\subsubsection{Single Visit to Reference Catalog, Semi-Blind}
\subsubsection{Multiple Visits to Reference Catalog}
\subsubsection{DIAObject Generation}
\subsubsection{Object Generation}
\subsubsection{Cross-Patch Merging}
\subsubsection{Cross-Tract Merging}
